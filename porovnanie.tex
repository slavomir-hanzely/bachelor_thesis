\chapter{Porovnanie metód}

Táto kapitola obsahuje praktické porovnania Metropolis--Hastings algoritmov spomenutých v prvej kapitole so zamietacou metódou pomocou MVEE elipsoidu. Porovnania boli naprogramované v jazyku Julia a spustené na počítači s procesorom Intel Core i5-6200U s frekvenciou 2.3GHz a RAM pamäťou veľkosti 8 GB pod systémom Linux.\\

Ako základ náhody budú náše generátory bodov v polyédri používať rovnomerný generátor čísel na $[-1,1]$ (ďalej $U[0,1]$). Pomocou $U[0,1]$ možno triviálne generovať bod na $[0,k]$ (prenásobením konštantou $k$), tiež možno generovať bod na $[a,b]$ (vygenerovaním bodu na $[0, -a+b]$ a pripočítaním konštanty $a$, alebo bod na $[0,1]^n$ (postupným vygenerovaním súradníc). Treba podotknúť, že generovanie na iných polyédroch, najmä v priestoroch vysokej dimenzie, je však vo všeobecnosti netriviálny problém.

Okrem generátora $U[0,1]$ budeme používať generátor z jednorozmerného normálneho rozdelenia $N(\mu, \sigma)$ so strednou hodnotou $\mu=0$ a smerodajnou odchýlkou $\sigma=1$. Vďaka rotačnej symetrickosti normálneho rozdelenia možno generovaním po zložkách pomocou $N(\mu, \sigma)$ získať $d$--rozmerné viacrozmerné normálne rozdelenie.
Keďže viacrozmerné normálne rozdelenie je centrálne symetrické, možno ho použiť na generovanie na $d$--rozmernej sfére ako bolo popísané v $\ref{generovanie_v_mvee}$.\\

Ako možné parametre porovnania možno uvažovať rýchlosť generovania bodov a ``rovnomernosť'' vygenerovaného rozdelenia (ako sa rozdelenie vygenerovaných bodov podobá rovnomernému rozdeleniu v polyédri).
Nakoľko v praxi pri generovaní malého počtu bodov nezáleží na efektivite generátora, možno použiť hocijakú zo spomenutých metód. Kedže tento prípad je v praxi nezaujímavý, v rámci testovania budeme porovnávať generovanie veľkého počtu bodov.

Pozrime sa na vierohodnosť vygenerovaných bodov. Zamietacie metódy, špeciálne aj MVEE metóda, generujú presne rovnomerné rozdelenia v polyédri. Metropolis--Hastings metódy sú v tomto smere nepresnejšie, generujú postupnosť, ktorá sa limitne blíži rovnomernému rozdeleniu v polyédri. Nakoľko uvažujeme generovanie veľkého počtu bodov, postupnosť vygenerovaných bodov sa bude aj pre Metropolis--Hastings metódy blížiť rovnemernému rozdeleniu v polyédri. Preto ak náhodne preusporiadame postupnosť takto vygenerovaných bodov, dostaneme rozdelenie bodov podobné rovnomernému rozdeleniu v polyédri, kde navyše po sebe idúce body takmer nie sú korelované (ako pri postupnosti priamo vygenerovaných bodov). Preto možno predpokladať, že vierohodnosť vygenerovanej postupnosti bodov bude v oboch prístupoch vysoká. Ďalej sa ňou nebudeme zaoberať, budeme porovnávať výlučne rýchlosť generovania veľkého počtu bodov. Taktiež, kedže nás zaujíma generovanie veľkého počtu bodov, do rýchlosti generovanie nebudeme zarátavať rýchlosť inicializácii algoritmov, v prípade zamietacej metódy pomocou MVEE nebudeme brať do úvahy čas potrebný na vypočítanie MVEE.\\

Pred spustením testovania sme predpokladali, že najrýchlejší generátor bude generátor pomocou zamietacej metódy využívajúcej MVEE, táto práca bude obsahovať implementáciu REX algoritmu ako rýchly nástroj na hľadanie MVEE (namiesto hľadania elipsoidu inou pomalšou, no jednoduchšou metódou).
